\documentclass{article}

\begin{document}

\title{Ethanol Precipitation of DNA}
\author{Liz Cooper}
\maketitle

\begin{enumerate}

\item Add 1/10th (of the sample volume) of 3M Sodium Acetate (pH 5.2) to the sample.  (i.e., if the volume of the DNA sample is 100 $\mu$L, add 10$\mu$L Sodium Acetate).

\item Add 1$\mu$L glycogen to each sample

\item Add 2.5-3X the volume of cold 95\% Ethanol.  (So, if the sample volume is 100$\mu$L, and you added 10$\mu$L Sodium Acetate, add 330$\mu$L EtOH).

\item Incubate on ice (or in the -20$^{\circ}$C freezer) for 15-60 minutes (if possible, use the full hour).

\item Centrifuge (at 4$^{\circ}$C) at 14,000g for 30 minutes.

\item Discard the supernatant, being careful not to throw out the DNA pellet.
	\begin{itemize}
		\item A good way to do this is to set up a stack of paper towels (about 1? high) beside the centrifuge, and then gently invert the tube or plate over the paper towels to let the ethanol pour off.  
		\item \textbf{Don't} tap or bang the upside down plate/tube at this point to try get all of the ethanol out; it is better to be left with a drop or two of ethanol than to accidentally dislodge your DNA pellet at this stage!
	\end{itemize}  

\item Add 700$\mu$L of 70\% Ethanol (cold EtOH is best) to each sample.

\item Centrifuge again for 15 minutes and then discard the supernatant (Same as above).

\item Repeat Steps 7 and 8.

\item After discarding the Ethanol from the 2nd wash, allow the pellet to air-dry for at least several hours, preferably overnight.  Then re-suspend the pellet in a suitable volume of the desired buffer.
	\begin{itemize}
		\item  It is ideal to do the precipitations in the mid to late afternoon, leave the samples to dry overnight, and then re-suspend them in the morning.
		\item To dry the samples, try to prop them up at an angle while they are upside down, to let the ethanol run off of the pellet right away, and then also to let air flow into the tube to help it dry completely.
	\end{itemize}
		
\end{enumerate}
		
		
\end{document}